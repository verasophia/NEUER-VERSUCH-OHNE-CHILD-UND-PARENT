% ____________________________________________________________________________
%	What's new?
% ____________________________________________________________________________
% ____________________________________________________________________________

%\documentclass[letterpaper,12pt]{article}
\documentclass[a4,12pt]{article}

% LaTeX Settings (fold)
\usepackage[utf8]{inputenc}

% Load Packages
%\usepackage{a4wide}%
\usepackage[top=3.0cm,left=3.0cm,right=3.0cm,bottom=3cm,headheight=1.5cm,headsep=1cm,footnotesep=1cm]{geometry}%
\usepackage{layout,fancyhdr,longtable,graphicx,float}%endfloat, wrapfig,shortvrb
\usepackage{setspace,natbib,booktabs,multirow,eurosym,lscape,tabularx,colortbl,fancyvrb,relsize,hvfloat}
\usepackage{fancyvrb,framed,rotating}%,ulem
\usepackage{pslatex}%
\usepackage[ngerman]{babel}
\usepackage[T1]{fontenc} %
%\usepackage[inline]{trackchanges}

\bibpunct{(}{)}{,}{a}{}{,}
%% Format URL
\usepackage{url}
%% Define a new 'leo' style for the package that will use a smaller font.
\makeatletter
\def\url@leostyle{%
  \@ifundefined{selectfont}{\def\UrlFont{\it}}{\def\UrlFont{\small\it}}}
\makeatother
%% Now actually use the newly defined style.
\urlstyle{leo}

% PSTricks
\usepackage{pstricks,pst-3dplot}%,pst-tree

\usepackage{amsfonts,amsmath,amssymb}
\newdimen\jot \jot=15pt
%\input{ueur.fd}
\DeclareMathVersion{euler} \DeclareMathVersion{eulerbold} \SetSymbolFont{letters}{euler}{U}{eur}{m}{n} \SetSymbolFont{letters}{eulerbold}{U}{eur}{b}{n} \SetSymbolFont{operators}{eulerbold}{OT1}{cmr}{bx}{n} \SetSymbolFont{symbols}{eulerbold}{OMS}{cmsy}{b}{n} \SetMathAlphabet\mathsf{eulerbold}{OT1}{cmss}{bx}{n} \SetMathAlphabet\mathit{eulerbold}{OT1}{cmr}{bx}{it}
\newcommand{\eqexcl}{\ensuremath{\stackrel{\mathrm{!}}{=}}}

\makeatletter
\def\eulermath{\@nomath\eulermath
              \mathversion{euler}}
\def\uneulermath{\@nomath\uneulermath
              \mathversion{normal}}
\def\eulerbfmath{\@nomath\eulerbfmath
              \mathversion{eulerbold}}
\def\uneulerbfmath{\@nomath\uneulerbfmath
              \mathversion{normal}}
\newcommand{\bfmath}[1]{\mbox{\boldmath$#1$\unboldmath}}
\newcommand{\eumath}[1]{\mbox{\eulermath$#1$\uneulermath}}
\newcommand{\eubmath}[1]{\mbox{\eulerbfmath$#1$\uneulerbfmath}}
\def\Var{\textrm{Var}}%
\def\Cov{\textrm{Cov}}%
\def\d{\textrm{d}}%
\makeatother%



% FORMATIERUNG

% Seitenumbruch
\clubpenalty = 10000 % schliesst Schusterjungen aus
\widowpenalty = 10000 % schliesst Hurenkinder aus

%Inhaltsverzeichnis
\usepackage{tocloft}
\renewcommand{\cfttoctitlefont}{}%\large\bf
\renewcommand{\cftloftitlefont}{\large\bf}
\renewcommand{\cftlottitlefont}{\large\bf}
%\renewcommand\cftsecfont{}
%\addtocontents{toc}{\hfill Seite\endgraf}
%\addtocontents{lof}{\ Abbildung\hfill Seite}
\renewcommand{\cftsecfont}{\normal} %  
%\renewcommand{\cftsecleader}{\normal\cftdotfill{\cftsecdotsep}} 
\renewcommand{\cftsecpagefont}{\normal} 
% \renewcommand{\cftsecpresnum}{SOMETHING } 
% \renewcommand{\cftsecaftersnum}{:} 
% \renewcommand{\cftsecaftersnumb}{\\}

%% List of Appendices
\newcommand\listappendixname{}
\newcommand\appcaption[1]{%
  \addcontentsline{app}{section}{#1}}
\makeatletter
\newcommand\listofappendices{%
  \section*{\listappendixname}\@starttoc{app}}
\makeatother


% PAGE-STYLE: Kopf- u. Fu{\ss}zeilen
%\addtolength{\parskip}{+0.5cm}%Paragraph Spacing // Absatz
%\parindent=0pt
\emergencystretch20pt%

% Page-Style: Fancy
%\renewcommand{\chaptermark}[1]{\markboth{#1}{}}

\renewcommand{\subsectionmark}[1]{\markright{\textit{\thesubsection\
#1}}{}}
\renewcommand{\sectionmark}[1]{\markright{\textit{\thesection.
#1}}}%
\addtolength{\headheight}{1pt} \lhead[\fancyplain{}{\thepage}] {\fancyplain{}{\rightmark}} \rhead[\fancyplain{}{\leftmark}] {\fancyplain{}{\thepage}}
\chead[\fancyplain{}{}]%
{\fancyplain{}{}} \cfoot{} \lfoot{\tiny }%\isodayandtime


% \"{U}BERSCHRIFTEN
\setcounter{secnumdepth}{3}% Tiefe der Nummerierung
\setcounter{tocdepth}{3}% Ebenen im TOC

\makeatletter
\renewcommand{\section}{%
    \@startsection {section}{1}{\z@}%
                   {-3.5ex plus -1ex minus -.2ex}%
                   {2.3ex plus.2ex}%
                   {\normalfont\normalsize\bfseries}}
\renewcommand{\subsection}{%
    \@startsection {subsection}{2}{\z@}%
                   {-3.5ex plus -1ex minus -.2ex}%
                   {2.3ex plus.2ex}%
                   {\normalfont\normalsize\bfseries}}
\renewcommand{\subsubsection}{%
    \@startsection {subsubsection}{3}{\z@}%
                   {-3.5ex plus -1ex minus -.2ex}%
                   {2.3ex plus.2ex}%
                   {\normalfont\normalsize\bfseries}}
\newcommand\myparagraph{\@startsection{paragraph}{4}{\z@}%
             {-3ex}%
             {0em}%
             {\reset@font\normalsize\textit}}
\newcommand\mysubparagraph{\@startsection{subparagraph}{5}{\z@}%
             {-3ex}%
             {-1em}%
             {\reset@font\normalsize}}

% Fu{\ss}noten
\usepackage[multiple]{footmisc}
\makeatletter
\newlength{\myFootnoteWidth}
\newlength{\myFootnoteLabel}
\setlength{\myFootnoteLabel}{1.2em}
\renewcommand{\@makefntext}[1]{%
  \setlength{\myFootnoteWidth}{\columnwidth}%
  \addtolength{\myFootnoteWidth}{-\myFootnoteLabel}%
  \noindent\makebox[\myFootnoteLabel][r]{\@makefnmark\ }%
  \parbox[t]{\myFootnoteWidth}{#1}%
}
\makeatother

% Datum und Uhrzeit
\newcount\m \newcount\n \begingroup
\count0=\time \divide\count0by60 % Hour
\count2=\count0 \multiply\count2by-60
\advance\count2by\time % Min
\def\2#1{\ifnum#1<10 0\fi\the#1}
\xdef\isodayandtime{\the\year-\2\month-\2
\day\space\2{\count0}:%
\2{\count2}} \endgroup


% Randbemerkungen
\let\margin\marginpar
\newcommand\myMargin[1]{\margin{\raggedright\renewcommand{\baselinestretch}{1}\footnotesize\textcolor{red}{#1}}}
\renewcommand{\marginpar}[1]{\myMargin{#1}}

%CAPTIONS
\usepackage{caption}[2004/11/28]
%\captionsetup{margin=0pt,font=normalsize,labelfont=,singlelinecheck=true,textfont={},belowskip=10pt}%labelsep=newline
\captionsetup{margin=0pt,font=normalsize,labelfont={bf},singlelinecheck=false,labelsep=newline,textfont={bf},belowskip=10pt}
\DeclareCaptionLabelFormat{simple}{#1 #2}%

\newenvironment{remark}{%
\renewcommand{\baselinestretch}{1.0}\normalsize
  \color{red}
  \raggedright
  \begin{framed}
  }
  {%
    \end{framed}%
    \color{black}
    \renewcommand{\baselinestretch}{1.5}\normalsize
}

\newtheorem{myhypo}{Hypothesis}


% LaTeX Settings (end)


\begin{document}
\pagestyle{plain}

\thispagestyle{empty}%
\begin{center}
\renewcommand{\baselinestretch}{1.5} %
\textbf{Projekt: Klasse im Puls -- Konzeptentwurf für eine Evaluation }\\[0.25cm]
\renewcommand{\baselinestretch}{1.5}\normalsize
Andrea Knecht, Gerhard Krug, Christoph Wunder\\ %$^\text{}$
Universität Erlangen-Nürnberg \\[0.15cm]
\today
\end{center}
\vspace{0.5cm}%
\noindent\renewcommand{\baselinestretch}{1.0}\normalsize %
% $^\text{a}$ University of Erlangen-Nuremberg \\
% $^\text{*}$ Corresponding author: Christoph Wunder, University of Erlangen-Nuremberg, Department of Economics, Lange Gasse 20, 90403 Nuremberg, Germany. Tel.: +49 911 5302 260; Fax: +49 911 5302 178. Email: christoph.wunder@wiso.uni-erlangen.de
% 
% \newpage

\section{Anhand welcher Auswirkungen soll das Projekt evaluiert werden?} % (fold)


\subsection{Nicht leistungsorientierte Auswirkungen}

	\begin{itemize}

		\item affirmative Grundeinstellung

			\begin{itemize}
				\item Zufriedenheit mit der schulischen Situation (NEPS \#6f) 

				\item Zeit für Hausaufgaben, Lernen

			\end{itemize}

		\item Integration insbesondere von Kindern mit Migrationshintergrund,
        hyperaktiven Schülern und Kindern mit sozialen Störungen
		
			\begin{itemize}

				\item Erhebung von Freundschaftsbeziehungen

				\item Hilfe von Mitschülern bei Hausaufgaben (NEPS \#22c)

				\item Migrationshintergrund 
				
					\begin{itemize}
					
					\item eigene Herkunft und Sprache (NEPS \#52-57)

					\item Herkunft und Sprache der Familie (NEPS \#15-19)

					\end{itemize}

				\item ,,Auffällige“ Schüler müssen evtl. über Lehrkraft
                identifiziert werden.
				
				 \end{itemize}
				
				 \item Freizeitverhalten (Sport, Musik), Lesen außerhalb der
                Schule

		\item abweichendes Verhalten (Gewaltbereitschaft)

		\item Selbstkonzept der Schüler/Softskills

			\begin{itemize}
		
			\item Zufriedenheit mit dem Leben, Gesundheit, Familie,
	              Bekannten- und Freundeskreis (NEPS \#6a-e)

			\item Einstellungsfragen (NEPS \#8) \\
			\includegraphics[scale=0.8]{../graphs/NEPS8.eps}

			\end{itemize}


	\end{itemize}
	

\subsection{Leistungsorientierte Auswirkungen}

	\begin{itemize}
		
		\item standardisierte Tests, Kompetenzen

		\item Noten im letzten Jahreszeugnis, aktuelle Noten,
        Selbsteinschätzung der schulischen Leistung

		\item sonderpädagogischer Förderbedarf

		\item Klassenwiederholung 

		\item Übergang in höhere Schule
		
		\item Selbsteinschätzung des Gesundheitszustandes (NEPS \#7)

	\end{itemize}

 
% section section_name (end)

\section{Methodische Ansätze} 

\subsection{\emph{Difference-in-differences} Ansatz} 

	\begin{itemize}

		\item Vergleich der Veränderung (der interessierenden Größe) in der
        Projekt-Klasse mit der Veränderung in einer Kontrollklasse. Es wird
        angenommen, dass sich die Schüler in beiden Klassen gleich entwickelt
        hätten, wenn das Projekt nicht durchgeführt worden wäre.

		\begin{center}
			\begin{tabular}{|l|c|c|}
				\hline
				& vorher 	& nachher \\
				\hline
				Projekt-Klasse & $y_{10}$ & $y_{11}$  \\
				\hline
				Kontrollklasse & $y_{00}$ & $y_{01}$ \\
				\hline
			\end{tabular}
		\end{center}
		
		$$ \text{Effekt} =  (y_{11} - y_{10}) - (y_{01} - y_{00}) $$

		\item erster Messzeitpunkt \underline{vor} Beginn des Projekts

        \item Messung der interessierenden Größen in einer Kontrollgruppe

	\end{itemize}

\subsection{Dynamische Netzwerkanalyse}

	\begin{itemize}

 		\item Beschreibung der Integration einer Schulklasse mittels
        dynamischer Netzwerkanalyse

		\item Bei neuer Klassenzusammensetzung sollten Messzeitpunkte
        (insbesondere für Netzwerk) in kurzen Abständen folgen ($<3$ Monate?).

	\end{itemize}

\subsection{Sonstiges}

	\begin{itemize}

 		\item Informationen zum Familienhintergrund (Bildung der Eltern)

		\item Information zur Schule (allgemeine Ausstattung, Stadt-Land,
        Ausländeranteil, Regionalinformation kann evtl. aus anderen Quellen
        zugespielt werden)

        \item Befragung durch Lehrer oder Interviewer? (Lehrer als Vertrauensperson ist u.~U. nicht neutral.)

        \item Pre-Test

        \item Datenschutz: keine Erhebung von Namen, Beziehungen werden über Ziffern kodiert.

        \item geschätzte Dauer: 1 Schulstunde

	\end{itemize}



% section methodischer_ansatz (end)

\newpage
% References (fold)	
\renewcommand{\baselinestretch}{1.0}\normalsize
\bibliographystyle{dcu}%apalikedcu
\addcontentsline{toc}{chapter}{\numberline{}\bibname}%
\bibliography{/Users/Christoph/Documents/Uni/Literatur/lit_DB/litdb_sp}
% (end)

\end{document}
