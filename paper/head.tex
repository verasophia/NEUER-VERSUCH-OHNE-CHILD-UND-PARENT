\usepackage[]{graphicx}
\usepackage[]{color}
%% maxwidth is the original width if it is less than linewidth
%% otherwise use linewidth (to make sure the graphics do not exceed the margin)
\makeatletter
\def\maxwidth{ %
  \ifdim\Gin@nat@width>\linewidth
    \linewidth
  \else
    \Gin@nat@width
  \fi
}
\makeatother

\definecolor{shadecolor}{rgb}{.97, .97, .97}
\definecolor{messagecolor}{rgb}{0, 0, 0}
\definecolor{warningcolor}{rgb}{1, 0, 1}
\definecolor{errorcolor}{rgb}{1, 0, 0}

\usepackage{alltt}

%\usepackage{stata}
\usepackage[utf8]{inputenc}
\usepackage{layout,fancyhdr,longtable,graphicx,float}%wrapfig,shortvrb
\usepackage{setspace,booktabs,multirow,lscape,tabularx,url,colortbl,fancyvrb,relsize,ltxtable, enumitem}
\usepackage{fancyvrb,framed,rotating}%,ulem

\usepackage{hyperref}
\usepackage{array}
\usepackage{wrapfig}
\usepackage{colortbl}
\usepackage{pdflscape}
\usepackage{tabu}
\usepackage{threeparttable} 
\usepackage{threeparttablex} 
\usepackage[normalem]{ulem} 
\usepackage{makecell}
\usepackage{xcolor}


\usepackage{pslatex}%
\usepackage[T1]{fontenc} %
\usepackage[english]{babel}
\usepackage{amsfonts,amsmath,amssymb}
\newdimen\jot \jot=15pt
\DeclareMathVersion{euler} \DeclareMathVersion{eulerbold} \SetSymbolFont{letters}{euler}{U}{eur}{m}{n} \SetSymbolFont{letters}{eulerbold}{U}{eur}{b}{n} \SetSymbolFont{operators}{eulerbold}{OT1}{cmr}{bx}{n} \SetSymbolFont{symbols}{eulerbold}{OMS}{cmsy}{b}{n} \SetMathAlphabet\mathsf{eulerbold}{OT1}{cmss}{bx}{n} \SetMathAlphabet\mathit{eulerbold}{OT1}{cmr}{bx}{it}
\newcommand{\eqexcl}{\ensuremath{\stackrel{\mathrm{!}}{=}}}

\makeatletter
\def\eulermath{\@nomath\eulermath
              \mathversion{euler}}
\def\uneulermath{\@nomath\uneulermath
              \mathversion{normal}}
\def\eulerbfmath{\@nomath\eulerbfmath
              \mathversion{eulerbold}}
\def\uneulerbfmath{\@nomath\uneulerbfmath
              \mathversion{normal}}
\newcommand{\bfmath}[1]{\mbox{\boldmath$#1$\unboldmath}}
\newcommand{\eumath}[1]{\mbox{\eulermath$#1$\uneulermath}}
\newcommand{\eubmath}[1]{\mbox{\eulerbfmath$#1$\uneulerbfmath}}
\def\Var{\textrm{Var}}%
\def\Cov{\textrm{Cov}}%
\def\d{\textrm{d}}%
\makeatother%

% Seitenumbruch
\clubpenalty = 10000 % schliesst Schusterjungen aus
\widowpenalty = 10000 % schliesst Hurenkinder aus

%Inhaltsverzeichnis
\usepackage{tocloft}
\renewcommand{\cfttoctitlefont}{}%\large\bf
\renewcommand{\cftloftitlefont}{\large\bf}
\renewcommand{\cftlottitlefont}{\large\bf}
%\renewcommand\cftsecfont{}
%\addtocontents{toc}{\hfill Seite\endgraf}
%\addtocontents{lof}{\ Abbildung\hfill Seite}
%\renewcommand{\cftsecfont}{\normal} %  
%\renewcommand{\cftsecleader}{\normal\cftdotfill{\cftsecdotsep}} 
%\renewcommand{\cftsecpagefont}{\normal} 
% \renewcommand{\cftsecpresnum}{SOMETHING } 
% \renewcommand{\cftsecaftersnum}{:} 
% \renewcommand{\cftsecaftersnumb}{\\}

%% List of Appendices
\newcommand\listappendixname{}
\newcommand\appcaption[1]{%
  \addcontentsline{app}{section}{#1}}
\makeatletter
\newcommand\listofappendices{%
  \section*{\listappendixname}\@starttoc{app}}
\makeatother


% PAGE-STYLE: Kopf- u. Fu{\ss}zeilen
\addtolength{\parskip}{+0.5cm}%Paragraph Spacing // Absatz
\parindent=0pt
\emergencystretch20pt%

% Page-Style: Fancy
\pagestyle{plain}
%\renewcommand{\chaptermark}[1]{\markboth{#1}{}}

\renewcommand{\subsectionmark}[1]{\markright{\textit{\thesubsection\
#1}}{}}
\renewcommand{\sectionmark}[1]{\markright{\textit{\thesection.
#1}}}%
\addtolength{\headheight}{1pt} \lhead[\fancyplain{}{\thepage}] {\fancyplain{}{\rightmark}} \rhead[\fancyplain{}{\leftmark}] {\fancyplain{}{\thepage}}
\chead[\fancyplain{}{}]%
{\fancyplain{}{}} \cfoot{} \lfoot{\tiny }%\isodayandtime

% Page-Style: Seite
\fancypagestyle{seite}{ \fancyhf{} \lhead[\fancyplain{}{\thepage}] {\fancyplain{}{XXX}} \rhead[\fancyplain{}{\leftmark}] {\fancyplain{}{\thepage}}
\chead[\fancyplain{}{}]%
{\fancyplain{}{}} \cfoot{} }

% Page-Style: Seite
\fancypagestyle{Landscape}{
\fancyhf{}
\lhead[\fancyplain{}{\thepage}]{\parbox[b]{14cm}{\fancyplain{}{\rightmark}}}
\rhead[\fancyplain{}{\leftmark}]{\fancyplain{}{\thepage}}
\chead[\fancyplain{}{}]{\fancyplain{}{}}
\cfoot{}
\lfoot{\tiny } %\isodayandtime
}


% \"{U}BERSCHRIFTEN
\setcounter{secnumdepth}{4}% Tiefe der Nummerierung
\setcounter{tocdepth}{4}% Ebenen im TOC

%% Section begins on new page 

\makeatletter
\renewcommand{\section}{%
    \@startsection {section}{1}{\z@}%
                   {-3.5ex plus -1ex minus -.2ex}%
                   {2.3ex plus.2ex}%
                   {\normalfont\normalsize\bfseries}}
\renewcommand{\subsection}{%
    \@startsection {subsection}{2}{\z@}%
                   {-3.5ex plus -1ex minus -.2ex}%
                   {2.3ex plus.2ex}%
                   {\normalfont\normalsize\bfseries}}
% \renewcommand{\subsubsection}{%
%     \@startsection {subsubsection}{3}{\z@}%
%                    {-3.5ex plus -1ex minus -.2ex}%
%                    {2.3ex plus.2ex}%
%                    {\normalfont\normalsize}}
\newcommand\subsubsubsection{\@startsection{paragraph}{4}{\z@}%
                   {-3.5ex plus -1ex minus -.2ex}%
                   {2.3ex plus.2ex}%
                   {\normalfont\normalsize\bfseries}}
\newcommand\mysubparagraph{\@startsection{subparagraph}{5}{\z@}%
             {-3ex}%
             {-1em}%
             {\reset@font\normalsize}}

\usepackage{titlesec}
% \newcommand{\sectionbreak}{\clearpage}
% Fu{\ss}noten
\usepackage[multiple]{footmisc}
\makeatletter
\newlength{\myFootnoteWidth}
\newlength{\myFootnoteLabel}
\setlength{\myFootnoteLabel}{1.2em}
\renewcommand{\@makefntext}[1]{%
  \setlength{\myFootnoteWidth}{\columnwidth}%
  \addtolength{\myFootnoteWidth}{-\myFootnoteLabel}%
  \noindent\makebox[\myFootnoteLabel][r]{\@makefnmark\ }%
  \parbox[t]{\myFootnoteWidth}{#1}%
}
\makeatother
\renewcommand{\footnotelayout}{\doublespacing}

% Datum und Uhrzeit
\newcount\m \newcount\n \begingroup
\count0=\time \divide\count0by60 % Hour
\count2=\count0 \multiply\count2by-60
\advance\count2by\time % Min
\def\2#1{\ifnum#1<10 0\fi\the#1}
\xdef\isodayandtime{\the\year-\2\month-\2
\day\space\2{\count0}:%
\2{\count2}} \endgroup


% Randbemerkungen
\let\margin\marginpar
\newcommand\myMargin[1]{\margin{\raggedright\renewcommand{\baselinestretch}{1}\footnotesize\textcolor{red}{#1}}}
\renewcommand{\marginpar}[1]{\myMargin{#1}}

%CAPTIONS
\usepackage{caption}[2004/11/28]
%\captionsetup{margin=0pt,font=normalsize,labelfont=,singlelinecheck=true,textfont={},belowskip=10pt}%labelsep=newline
\captionsetup{margin=0pt,font=normalsize,labelfont={bf},singlelinecheck=false,labelsep=newline,textfont={bf},belowskip=10pt}
\DeclareCaptionLabelFormat{simple}{#1 #2}%
\IfFileExists{upquote.sty}{\usepackage{upquote}}{}

%%%%%%%%%%%%%%%%%%%%%%%%%%%%%%%%%%%%%%%%%%%%%%%%%%%%%%%%%%%%%%%%%%%%%%%%%%%%%%%%
%   COLORED LISTING
%   Source: https://tex.stackexchange.com/questions/163412/how-fancyvrb-background-color-fill-completely-with-fillcolor
%%%%%%%%%%%%%%%%%%%%%%%%%%%%%%%%%%%%%%%%%%%%%%%%%%%%%%%%%%%%%%%%%%%%%%%%%%%%%%%%

\usepackage{courier}
\usepackage{tcolorbox,listings}
\lstdefinestyle{mystyle}{
     basicstyle=\relsize{-2.5}\ttfamily, 
     numbers=none, 
     numberstyle=\tiny, 
     numbersep=0pt,
     showspaces=false,
     breaklines=false
 }
\tcbuselibrary{listings,skins,breakable}
\definecolor{light-gray}{gray}{0.95}
\definecolor{mid-gray}{gray}{0.5}
\newtcblisting{BGVerbatim}{
      arc=0mm,
      top=-2mm,
      bottom=-2mm,
      left=0mm,
      right=0mm,
      width=\linewidth,
      boxrule=0.5,
      colback=light-gray,
      spartan,
      listing only,
      listing options={style=mystyle},
      breaklines=false
}

\RecustomVerbatimCommand{\VerbatimInput}{VerbatimInput}%
{fontsize=\footnotesize,
 %
 frame=lines,  % top and bottom rule only
 framesep=2em, % separation between frame and text
 % rulecolor=\color{mid-gray},
 %
 % label=\fbox{\color{Black}data.txt},
 % labelposition=topline,
 %
 % commandchars=\|\(\), % escape character and argument delimiters for
                      % commands within the verbatim
 % commentchar=*        % comment character
}

